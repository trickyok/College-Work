\documentclass[14pt]{extarticle}
\usepackage{amssymb,amsthm,amsmath, color}
\usepackage{graphicx}
\usepackage{float}
\usepackage{fullpage}
\usepackage{subfigure}
\usepackage{graphics}
\usepackage{mdframed}
\usepackage{multicol}

\usepackage{listings, multicol}
\usepackage{xcolor} % for setting colors

% set the default code style
\lstset{
    %frame=single, % draw a frame at the top and bottom of the code block
    tabsize=4, % tab space width
    showstringspaces=false, % don't mark spaces in strings
    %numbers=left, % display line numbers on the left
    commentstyle=\color{green}, % comment color
    keywordstyle=\color{blue}, % keyword color
    stringstyle=\color{red} % string color
}

\newtheorem{theorem}{Theorem}[section]
\newtheorem{lemma}[theorem]{Lemma}
\newtheorem{proposition}[theorem]{Proposition}
\newtheorem{claim}[theorem]{Claim}
\newtheorem{corollary}[theorem]{Corollary}
\newtheorem{definition}[theorem]{Definition}
\newtheorem{observation}[theorem]{Observation}
\newtheorem{fact}[theorem]{Fact}
\newtheorem{property}{Property}
\newtheorem{remark}{Remark}[section]
\newtheorem{notation}{Notation}[section]
\newtheorem{example}{Example}[section]
\newtheorem{algorithm}{Algorithm}
\newtheorem{conjecture}{Conjecture}
\newtheorem{question}[conjecture]{Question}

\newcommand\tab[1][1cm]{\hspace*{#1}}

\begin{document}
\textbf{CSE 2321 Homework 7}

\section*{Problem 1}

% You don't need to write proofs here, just give the expression and evalaute it. The align environment is good for this:
% The '&' tells it what to align each line on
% The '\\' tells it where to make a new line
% The '&&' and '\text{}' is good for adding comments to a line, I have an example below
\begin{align*}
T(n) &= \sum_{k=0}^{n-1} ar^k && \text{Geometric series is really useful!}  \\
&= a \left( \frac{1 - r^n}{1 - r} \right) && \text{If } r \neq 1
\end{align*}

\subsection*{1A}

Recurrence relation:
$$T(n) = 2T\left(\frac{n}{2}\right) + 1$$
Running time:
$$T(n) = \Theta(n)$$

Generic formula after k substitutions:
\begin{equation}
T(n) = 2^{k+1}T\left(\frac{n}{2^{k+1}}\right) + \sum_{i=0}^{k} 2^i
\end{equation}
$$\text{Stop when } \frac{n}{2^{k+1}} = 1$$
$$k = \log_2(n) - 1$$
\begin{equation}
T(n) = 2^{(\log_2(n) - 1 + 1)}T\left(\frac{n}{2^{(\log_2(n) - 1 + 1)}}\right) + \sum_{i=0}^{(\log_2(n) - 1)} 2^i
\end{equation}
$$n \cdot 1 + n - 1$$
$$T(n) = \Theta(n)$$

\subsection*{1B}
Recurrence relation:
$$T(n) = 3T\left(\frac{n}{3}\right) + 1$$
Running time:
$$T(n) = \Theta(n)$$
Generic formula after k substitutions:
\begin{equation}
T(n) = 3^{k+1}T\left(\frac{n}{3^{k+1}}\right) + \sum_{i=0}^{k} 3^i
\end{equation}
$$\text{Stop when } \frac{n}{3^{k+1}} = 1$$
$$k = \log_3(n) - 1$$
\begin{equation}
T(n) = 3^{(\log_3(n) - 1 + 1)}T\left(\frac{n}{3^{(\log_3(n) - 1 + 1)}}\right) + \sum_{i=0}^{(\log_3(n) - 1)} 3^i
\end{equation}
$$n \cdot 1 + \frac{n-1}{2}$$
$$T(n) = \Theta(n)$$
\subsection*{1C}
Recurrence relation:
$$T(n) = T(n-1) + 1$$
Running time:
$$T(n) = \Theta(n)$$
Generic formula after k substitutions:
\begin{equation}
T(n) = T(n-(k+1)) + (k+1)
\end{equation}
$$\text{Stop when } n - (k+1) = 1$$
$$k = n-2$$
\begin{equation}
T(n - ((n - 2) + 1)) + ((n - 2) + 1)
\end{equation}
$$1 + n - 1 = n$$
$$T(n) = \Theta(n)$$
\subsection*{1D}
Recurrence relation:
$$T(n) = T(n-2) + 1$$
Running time:
$$T(n) = \Theta(n)$$
Generic formula after k substitutions:
\begin{equation}
T(n) = T(n-2(k+1)) + (k+1)
\end{equation}
$$\text{Stop when } n-2(k+1)) = 1$$
$$k = \frac{n-3}{2}$$
$$T(n) = T\left(n - 2\left(\frac{n-3}{2}+1\right)\right) + \left(\frac{n-3}{2} + 1\right)$$
$$1 + \frac{n-3}{2} + 1$$
$$T(n) = \Theta(n)$$
\subsection*{1E}
Recurrence relation:
$$T(n) = 3T\left(\frac{n}{2}\right) + 1$$
Running time:
$$T(n) = \Theta(n^{\frac{\log_3(n)}{\log_2(n)}})$$
Generic formula after k substitutions:
\begin{equation}
T(n) = 3^{k+1}T\left(\frac{n}{2^{k+1}}\right) + \sum_{i=0}^{k} 3^i
\end{equation}
$$\text{Stop when } \frac{n}{2^{k+1}} = 1$$
$$k = \log_2(n) - 1$$
\begin{equation}
T(n) = 3^{(\log_2(n) - 1 + 1)}T\left(\frac{n}{2^{(\log_2(n) - 1 + 1)}}\right) + \sum_{i=0}^{(\log_2(n) - 1)} 3^i
\end{equation}
$$3^{\log_2(n)} \cdot 1 + \frac{{1 - 3^{\log_2(n)}}}{-2}$$
$$T(n) = \Theta(n^{\frac{\log_3(n)}{\log_2(n)}})$$




\section*{Problem 3}
% For your version of trinary search, you can use the lstlisting environment like I have for BinarySearch, just replace it with your code:
\begin{lstlisting}[language=Python]
int BinarySearch(A, i, j, k)
	if i>j
		index = -1
	else
		midpt1 = (i+j)/3
		midpt2 = midpt1
		if k = A[midpt1]
			index = midpt1
		else if k < A[midpt1]
			index = BinarySearch(A, i, midpt1 - 1, k)
		else if k = A[midpt2]
			index = midpt2
		else if k > A[midpt2]
			index = BinarySearch(A, midpt2 + 1, j, k)
		else
			index = BinarySearch(A, midpt1 + 1, midpt2 - 1, k)
	return index
\end{lstlisting}
% For you analysis of the running time, you can again use the align* argument from before.

TrinarySearch: \\
Recurrence relation:
$$T(n) = T\left(\frac{n}{3}\right) + C$$
Running time:
$$T(n) = \Theta(log(n))$$ \\ \\
Generic formula after k substitutions:
\begin{equation}
T(n) = T\left(\frac{n}{3^{k+1}}\right) + (k + 1)
\end{equation}
$$\text{Stop when } \frac{n}{3^{k+1}} = 1$$
$$k = \log_3(n) - 1$$
\begin{equation}
T(n) = T\left(\frac{n}{3^{(\log_3(n) - 1 + 1)}}\right) + (\log_3(n) - 1 + 1)
\end{equation}
$$1 + log_3(n)$$
$$T(n) = \Theta(log(n))$$


BinarySearch (from class notes): \\ 
Recurrence relation:
$$T(n) = T\left(\frac{n}{2}\right) + C$$
Running time:
$$T(n) = \Theta(log(n))$$
TrinarySearch has same running time as BinarySearch.




\end{document}
