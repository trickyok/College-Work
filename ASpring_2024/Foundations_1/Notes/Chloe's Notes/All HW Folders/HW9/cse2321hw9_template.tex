\documentclass[14pt]{extarticle}
\usepackage{amssymb,amsthm,amsmath, color}
\usepackage{graphicx}
\usepackage{float}
\usepackage{fullpage}
\usepackage{subfigure}
\usepackage{graphics}
\usepackage{mdframed}
\usepackage{multicol}

\usepackage{listings, multicol}
\usepackage{xcolor} % for setting colors

% set the default code style
\lstset{
    %frame=single, % draw a frame at the top and bottom of the code block
    tabsize=4, % tab space width
    showstringspaces=false, % don't mark spaces in strings
    %numbers=left, % display line numbers on the left
    commentstyle=\color{green}, % comment color
    keywordstyle=\color{blue}, % keyword color
    stringstyle=\color{red} % string color
}

\newtheorem{theorem}{Theorem}[section]
\newtheorem{lemma}[theorem]{Lemma}
\newtheorem{proposition}[theorem]{Proposition}
\newtheorem{claim}[theorem]{Claim}
\newtheorem{corollary}[theorem]{Corollary}
\newtheorem{definition}[theorem]{Definition}
\newtheorem{observation}[theorem]{Observation}
\newtheorem{fact}[theorem]{Fact}
\newtheorem{property}{Property}
\newtheorem{remark}{Remark}[section]
\newtheorem{notation}{Notation}[section]
\newtheorem{example}{Example}[section]
\newtheorem{algorithm}{Algorithm}
\newtheorem{conjecture}{Conjecture}
\newtheorem{question}[conjecture]{Question}

\newcommand\tab[1][1cm]{\hspace*{#1}}

\begin{document}
\textbf{CSE 2321 Homework 9 Template}
\section*{Problem 1}

\begin{lstlisting}[language=Python]
DFSModified(G)
	boolean hamiltonian = false
	node s = u # s is the start node
	for each u in V
		u.color = white
		u.p = nil	# p is for parent
	for each u in V
		if u.color == white
			hamiltonian = Visit(G,u, s)
		if u.color == black
			u.color == white # revisit past nodes
	return hamiltonian
\end{lstlisting}

\begin{lstlisting}[language=Python]
Visit(G, u, s)
	boolean hamiltonian = false
	for each v in G.Adj[u]
		u.color = gray
		if v.color == white
		v.p = u
		Visit(G, v)
		u.color = black
		if v == s # If the next node is the starting node
			hamiltonian = true
\end{lstlisting}

\section*{Problem 2}
\begin{lstlisting}[language=Python]
FindSink(G) 
	int i, j = 0
	boolean hasSink = false
	for i to |V| && !hasSink
		if !(i == j) && G[i][j] == 0 
		# j is not a sink, next column
			j++;
			hasSink = isSink(i)
		else if !(i == j) && G[i][j] == 1 
		 # i is not a sink, next row and column
			i++
			j++
			hasSink = isSink(j)
		else if i > |V| || j > |V|
			hasSink = false
	return hasSink
\end{lstlisting}

\begin{lstlisting}[language=Python]
isSink(G, x) 
	boolean isSink = true
	
	for int i = 0 to |V|
		# The xth row should be all 0
		if G[k][i] != 0
			isSink = false
		i++
	for int i = 0 to |V|
		# The xth column should be all 1s
		if i != k && G[i][k] != 1 
			isSink = false
		else if !(i == j) && G[i][j] == 1 
			isSink == false
		i++	
	return isSink
\end{lstlisting}

\section*{Problem 3}
	\begin{lstlisting}[language=Python]
Diameter(G)
	int max = 0 # the diameter
	for int i = 0 to |V| # test all starting points
		# check all distances to other vertices
		for int j = 0 to |V| 
			int length = BFSModified(G, i)
			# Note: if two nodes are not connected, infinity
			if length != infinity && length > max 
				max = length
			j++
		i++
	
	return max
\end{lstlisting}

\begin{lstlisting}[language=Python]
BFSModified(G, s, e)
	for each v in V
		v.dist = infinity
		s.dist = 0
		q.enqueue(s)
		while q != 0
			v.dequeue
				for each w s.t. (v,w) in E
					if w.dist = infinity
						q.enqueue(w)
	return w.dist # Modified to return length of path
\end{lstlisting}

The running time of this algorithm is O($|V|^2 + |V||E|$).
BFSModified has the same running time as BFS, which is O($|V|+|E|$).
The for loops have the running time of O($|V|^2$).

\end{document}
