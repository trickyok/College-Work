\documentclass[12 pt]{amsart}

\usepackage{amsmath, amsfonts, amssymb,amsthm}
\usepackage[pdftex]{color,graphicx}
\usepackage[margin=1 in]{geometry}
\usepackage[all]{xy}
\usepackage{bbm}
\usepackage{mathrsfs}
%\usepackage{enumerate}
\usepackage{enumitem}
\usepackage[all]{xy}
\usepackage{tasks}

\setlength{\parindent}{2 em}
\setlength{\parskip}{0.2 ex}
\linespread{1.2}

\newcommand{\Rr}{\mathbb{R}}
\newcommand{\Qq}{\mathbb{Q}}
\newcommand{\Zz}{\mathbb{Z}}
\newcommand{\Nn}{\mathbb{N}}
\newcommand{\Cc}{\mathbb{C}}
\newcommand{\Tt}{\mathbb{T}}
\newcommand{\Ss}{\mathbb{S}}
\newcommand{\Dd}{\mathbb{D}}
\newcommand{\Rp}{\mathbb{R}\mathrm{P}}
\newcommand{\tH}{\widetilde{H}}
\newcommand{\id}{\mathbbm{1}}
\newcommand{\PWR}{\mathscr{P}}

\newcommand{\sse}{\subseteq}
\newcommand{\lra}{\Leftrightarrow}
\newcommand{\imp}{\Rightarrow}

\newcommand{\toup}[1]{\xrightarrow{#1}}
\newcommand{\mc}[1]{\mathcal{#1}}
\newcommand{\mb}[1]{\mathbf{#1}}

\newcommand{\vs}{\vspace{\stretch{1}}}

\newcommand{\hs}{\hspace*{0.5cm}}

%\renewcommand{\theenumi}{(\alph{enumi})}
\renewcommand{\epsilon}{\varepsilon}
\renewcommand{\phi}{\varphi}

\DeclareMathOperator{\im}{Im}
\DeclareMathOperator{\sgn}{sgn}
\DeclareMathOperator{\ad}{ad}
\DeclareMathOperator{\Char}{char}
\DeclareMathOperator{\Aut}{Aut}
\DeclareMathOperator{\End}{End}
\DeclareMathOperator{\tr}{tr}
\DeclareMathOperator{\Rad}{Rad}
\DeclareMathOperator{\dom}{dom}
\DeclareMathOperator{\cod}{cod}
\DeclareMathOperator{\nat}{Nat}
\DeclareMathOperator{\POW}{POW}

\newcommand{\st}{such that}
\newcommand{\tfae}{the following are equivalent}
\newcommand{\Tfae}{The following are equivalent}
\newcommand{\ifff}{if and only if}

\newtheorem{thm}{Theorem}
\newtheorem{lemma}[thm]{Lemma}
\newtheorem{df}[thm]{Definition}
\newtheorem{prop}[thm]{Proposition}
\newtheorem*{exam}{Example(s)}
\newtheorem*{claim}{Claim}

\begin{document}

\begin{center} \Large
	CSE 2321 \quad Foundations I \quad Spring, 2024 \quad Dr. Estill \\ \large
	Homework 11 \quad Due: Friday, April 19
\end{center}

\vspace{3ex}


\begin{enumerate}[label=\arabic*.)]
	
	\item (30 points) Use the first topological sort algorithm (``$TopSort$'') to topologically sort the following acyclic graph.
	Assume that the vertex list and all adjacency lists are in alphabetical order.
	
	\[\xymatrix@+20pt{
		*++[o][F]{A} \ar[ddr] 				& *++[o][F]{C} \ar[r] \ar[d] \ar[l]	 	& *++[o][F]{E} \ar[dll] \\
		*++[o][F]{G} \ar[drr] \ar[u]			& *++[o][F]{H} 					& *++[o][F]{I} \ar[ul] \ar[u] \ar[d] \\
		*++[o][F]{N} \ar[urr] \ar[u] \ar[r]		& *++[o][F]{P} \ar[u]				& *++[o][F]{R} \ar@/_2pc/[uull]
	}\]
	
	\vspace{1.5cm}
	
	Write down the sorted vertices: \rule{3 in}{0.1pt}
	
	\newpage
	
	\noindent We'll be finding the strongly-connected components (SCCs) of the following graph.
	
	\vspace{1cm}
	
	\item (25 points) Run the depth-first search algorithm on the following graph.
	Room has been left in the vertices for the discovery and finish times (which are required) and the predecessor (which isn't required, but is recommended).
	
	\noindent\emph{Note: Read question 3 before starting the work.}
	
	\smallskip
	
	\begin{center}
		\includegraphics[scale=1]{SCCGraph}
	\end{center}
	
	\bigskip
	
	\item (10 points) List the vertices in reverse finish time order (i.e., from last finished to first).
	%\rule{5.5 in}{0.1pt}
	
	\bigskip
	
	\pagebreak
	
	\item (25 points) Run the depth-first search algorithm on the following graph with the vertex list and all adjacency lists in the order given by the previous part.
	Mark tree edges and no other type of edges.
	No other information is required, but use what will help you keep track.
	
	\smallskip
	
	\begin{center}
		\includegraphics[scale=1]{SCCGraphTranspose}
	\end{center}
	
	\bigskip
	
	\item (10 points) List the strongly connected components as sets or circle them on the cleaner of the two above graphs.

\end{enumerate}


\end{document}
