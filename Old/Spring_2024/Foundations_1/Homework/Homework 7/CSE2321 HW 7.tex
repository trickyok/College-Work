\documentclass[12 pt]{amsart}

\usepackage{amsmath, amsfonts, amssymb,amsthm}
\usepackage[pdftex]{color,graphicx}
\usepackage[margin=1 in]{geometry}
\usepackage[all]{xy}
\usepackage{bbm}
\usepackage{mathrsfs}
%\usepackage{enumerate}
\usepackage{enumitem}
\usepackage[all]{xy}
\usepackage{tasks}

\setlength{\parindent}{2 em}
\setlength{\parskip}{0.2 ex}
\linespread{1.2}

\newcommand{\Rr}{\mathbb{R}}
\newcommand{\Qq}{\mathbb{Q}}
\newcommand{\Zz}{\mathbb{Z}}
\newcommand{\Nn}{\mathbb{N}}
\newcommand{\Cc}{\mathbb{C}}
\newcommand{\Tt}{\mathbb{T}}
\newcommand{\Ss}{\mathbb{S}}
\newcommand{\Dd}{\mathbb{D}}
\newcommand{\Rp}{\mathbb{R}\mathrm{P}}
\newcommand{\tH}{\widetilde{H}}
\newcommand{\id}{\mathbbm{1}}
\newcommand{\PWR}{\mathscr{P}}

\newcommand{\sse}{\subseteq}
\newcommand{\lra}{\Leftrightarrow}
\newcommand{\imp}{\Rightarrow}

\newcommand{\toup}[1]{\xrightarrow{#1}}
\newcommand{\mc}[1]{\mathcal{#1}}
\newcommand{\mb}[1]{\mathbf{#1}}

\newcommand{\vs}{\vspace{\stretch{1}}}

\newcommand{\hs}{\hspace*{0.5cm}}

%\renewcommand{\theenumi}{(\alph{enumi})}
\renewcommand{\epsilon}{\varepsilon}
\renewcommand{\phi}{\varphi}

\DeclareMathOperator{\im}{Im}
\DeclareMathOperator{\sgn}{sgn}
\DeclareMathOperator{\ad}{ad}
\DeclareMathOperator{\Char}{char}
\DeclareMathOperator{\Aut}{Aut}
\DeclareMathOperator{\End}{End}
\DeclareMathOperator{\tr}{tr}
\DeclareMathOperator{\Rad}{Rad}
\DeclareMathOperator{\dom}{dom}
\DeclareMathOperator{\cod}{cod}
\DeclareMathOperator{\nat}{Nat}
\DeclareMathOperator{\POW}{POW}

\newcommand{\st}{such that}
\newcommand{\tfae}{the following are equivalent}
\newcommand{\Tfae}{The following are equivalent}
\newcommand{\ifff}{if and only if}

\newtheorem{thm}{Theorem}
\newtheorem{lemma}[thm]{Lemma}
\newtheorem{df}[thm]{Definition}
\newtheorem{prop}[thm]{Proposition}
\newtheorem*{exam}{Example(s)}
\newtheorem*{claim}{Claim}

\begin{document}

\begin{center} \Large
	CSE 2321 \quad Foundations I \quad Spring, 2024 \quad Dr. Estill \\ \large
	Homework 7 \quad Due: Friday, March 22
\end{center}

\vspace{3ex}

\begin{thm}[Master Theorem] \small
	\rule{0pt}{1ex} \\ Let $a \geq 1$ and $b>1$ be constants, let $f(n)$ be a function $f:\Nn \to \Rr^+$, and let $T(n)$ be defined on the nonnegative integers by the recurrence $$T(n) = aT(n/b) + f(n).$$
	Then $T(n)$ has the following asymptotic bounds:
	\begin{enumerate}[label=\arabic*.]
		\item if $f(n) \in O(n^{\log_b a - \epsilon})$ for some constant $\epsilon > 0$, then $T(n) \in \Theta(n^{\log_b a})$,
		\item if $f(n) \in \Theta(n^{\log_b a})$ then $T(n) \in \Theta(n^{\log_b a}\log n)$, and
		\item if $f(n) \in \Omega(n^{\log_b a+ \epsilon})$ for some $\epsilon>0$ and if $a f(n/b) \leq d f(n)$ for some constant $d<1$ and all sufficiently large n, then $T(n) \in \Theta(f(n))$.
	\end{enumerate}
\end{thm}

\bigskip

Use the Master Theorem above to solve the following recurrences when possible.
If you need to confirm the regularity condition (\(af(n/b) \leq df(n)\) for \(d < 1\)), work should be shown, but otherwise answers are all that is needed.
(Note that not every blank needs to be filled in in every problem.):

\emph{(20 points each)}

\bigskip

\begin{enumerate}[label=\arabic*.)]

%\begin{enumerate}[label=\alph*.)]
	\item $T(n) = T(n/3) + c$ \\
	$f(n) = \underline{\phantom{xxxxxxxx}}$ versus $n^{\log_b a} = \underline{\phantom{xxxxxxxx}}$ \\
	Which is growing faster: $f(n)$ or  $n^{\log_b a}$?  \underline{\phantom{xxxxxxxxxxx}} \\
	Which case of the Master Theorem does that potentially put us in?  \underline{\phantom{xxxx}} \\
	If you're potentially in case one or three, is it possible to find an epsilon which makes either $f(n) \in O(n^{\log_b a - \epsilon})$ (if you're in case one) or $f(n) \in \Omega(n^{\log_b a+ \epsilon})$ (if you're in case three) true? Choose one or show an inequality. \underline{\phantom{xxxxx}} \\
	If you're potentially in case three and there is an $\epsilon$, try to find a constant $d<1$ such that $a f(n/b) \leq d f(n)$ for large enough $n$'s.
	\underline{\phantom{xxxxxxxxxxxxxxx}}\\
	What can you conclude? \underline{\phantom{xxxxxxxxxxxxxxx}}
	
	\bigskip
	
	\item $T(n) = T(n/3) + c\log_2 n$ \\
	$f(n) = \underline{\phantom{xxxxxxxx}}$ versus $n^{\log_b a} = \underline{\phantom{xxxxxxxx}}$ \\
	Which is growing faster: $f(n)$ or  $n^{\log_b a}$?  \underline{\phantom{xxxxxxxxxxx}} \\
	Which case of the Master Theorem does that potentially put us in?  \underline{\phantom{xxxx}} \\
	If you're potentially in case one or three, is it possible to find an epsilon which makes either $f(n) \in O(n^{\log_b a - \epsilon})$ (if you're in case one) or $f(n) \in \Omega(n^{\log_b a+ \epsilon})$ (if you're in case three) true? Choose one or show an inequality. \underline{\phantom{xxxxx}} \\
	If you're potentially in case three and there is an $\epsilon$, try to find a constant $d<1$ such that $a f(n/b) \leq d f(n)$ for large enough $n$'s.
	\underline{\phantom{xxxxxxxxxxxxxxx}}\\
	What can you conclude? \underline{\phantom{xxxxxxxxxxxxxxx}}
	
	\bigskip
	
	\item $T(n) = 4T(n/2) + cn$ \\
	$f(n) = \underline{\phantom{xxxxxxxx}}$ versus $n^{\log_b a} = \underline{\phantom{xxxxxxxx}}$ \\
	Which is growing faster: $f(n)$ or  $n^{\log_b a}$?  \underline{\phantom{xxxxxxxxxxx}} \\
	Which case of the Master Theorem does that potentially put us in?  \underline{\phantom{xxxx}} \\
	If you're potentially in case one or three, is it possible to find an epsilon which makes either $f(n) \in O(n^{\log_b a - \epsilon})$ (if you're in case one) or $f(n) \in \Omega(n^{\log_b a+ \epsilon})$ (if you're in case three) true? Choose one or show an inequality. \underline{\phantom{xxxxx}} \\
	If you're potentially in case three and there is an $\epsilon$, try to find a constant $d<1$ such that $a f(n/b) \leq d f(n)$ for large enough $n$'s.
	\underline{\phantom{xxxxxxxxxxxxxxx}}\\
	What can you conclude? \underline{\phantom{xxxxxxxxxxxxxxx}}
	
	\bigskip
	
	\item $T(n) = 4T(n/2) + cn^3$ \\
	$f(n) = \underline{\phantom{xxxxxxxx}}$ versus $n^{\log_b a} = \underline{\phantom{xxxxxxxx}}$ \\
	Which is growing faster: $f(n)$ or  $n^{\log_b a}$?  \underline{\phantom{xxxxxxxxxxx}} \\
	Which case of the Master Theorem does that potentially put us in?  \underline{\phantom{xxxx}} \\
	If you're potentially in case one or three, is it possible to find an epsilon which makes either $f(n) \in O(n^{\log_b a - \epsilon})$ (if you're in case one) or $f(n) \in \Omega(n^{\log_b a+ \epsilon})$ (if you're in case three) true? Choose one or show an inequality. \underline{\phantom{xxxxx}} \\
	If you're potentially in case three and there is an $\epsilon$, try to find a constant $d<1$ such that $a f(n/b) \leq d f(n)$ for large enough $n$'s.
	\underline{\phantom{xxxxxxxxxxxxxxx}}\\
	What can you conclude? \underline{\phantom{xxxxxxxxxxxxxxx}}
	
	\bigskip
	
	\item $T(n) = 2T(n/6) + \sqrt{n\,}$ \\
	$f(n) = \underline{\phantom{xxxxxxxx}}$ versus $n^{\log_b a} = \underline{\phantom{xxxxxxxx}}$ \\
	Which is growing faster: $f(n)$ or  $n^{\log_b a}$?  \underline{\phantom{xxxxxxxxxxx}} \\
	Which case of the Master Theorem does that potentially put us in?  \underline{\phantom{xxxx}} \\
	If you're potentially in case one or three, is it possible to find an epsilon which makes either $f(n) \in O(n^{\log_b a - \epsilon})$ (if you're in case one) or $f(n) \in \Omega(n^{\log_b a+ \epsilon})$ (if you're in case three) true? Choose one or show an inequality. \underline{\phantom{xxxxx}} \\
	If you're potentially in case three and there is an $\epsilon$, try to find a constant $d<1$ such that $a f(n/b) \leq d f(n)$ for large enough $n$'s.
	\underline{\phantom{xxxxxxxxxxxxxxx}}\\
	What can you conclude? \underline{\phantom{xxxxxxxxxxxxxxx}}
	
\end{enumerate}


\end{document}
