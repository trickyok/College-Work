\documentclass[14pt]{extarticle}
\usepackage{amssymb,amsthm,amsmath, color}
\usepackage{graphicx}
\usepackage{float}
\usepackage{fullpage}
\usepackage{subfigure}
\usepackage{graphics}
\usepackage{mdframed}
\usepackage{multicol}

\usepackage{listings, multicol}
\usepackage{xcolor} % for setting colors

% set the default code style
\lstset{
    %frame=single, % draw a frame at the top and bottom of the code block
    tabsize=4, % tab space width
    showstringspaces=false, % don't mark spaces in strings
    %numbers=left, % display line numbers on the left
    commentstyle=\color{green}, % comment color
    keywordstyle=\color{blue}, % keyword color
    stringstyle=\color{red} % string color
}

\newtheorem{theorem}{Theorem}[section]
\newtheorem{lemma}[theorem]{Lemma}
\newtheorem{proposition}[theorem]{Proposition}
\newtheorem{claim}[theorem]{Claim}
\newtheorem{corollary}[theorem]{Corollary}
\newtheorem{definition}[theorem]{Definition}
\newtheorem{observation}[theorem]{Observation}
\newtheorem{fact}[theorem]{Fact}
\newtheorem{property}{Property}
\newtheorem{remark}{Remark}[section]
\newtheorem{notation}{Notation}[section]
\newtheorem{example}{Example}[section]
\newtheorem{algorithm}{Algorithm}
\newtheorem{conjecture}{Conjecture}
\newtheorem{question}[conjecture]{Question}

\newcommand\tab[1][1cm]{\hspace*{#1}}

\begin{document}
\textbf{CSE 2321 Homework 6}

\section*{Problem 1}

% You don't need to write proofs here, just give the expression and evalaute it. The align environment is good for this:
% The '&' tells it what to align each line on
% The '\\' tells it where to make a new line
% The '&&' and '\text{}' is good for adding comments to a line, I have an example below
\begin{align*}
T(n) &= \sum_{k=0}^{n-1} ar^k && \text{Geometric series is really useful!}  \\
&= a \left( \frac{1 - r^n}{1 - r} \right) && \text{If } r \neq 1
\end{align*}

\subsection*{1A}
\begin{align*}
T(n) &= \sum_{a=1}^{n^2}(\sum_{b=1}^{n^2}(\sum_{c=1}^{b}1)) && \text{}  \\
&= \sum_{a=1}^{n^2}(\sum_{b=1}^{n^2}b) && \text{} \\
&= \sum_{a=1}^{n^2}(\frac{n^2(n^2+1)}{2}) && \text{} \\
&= \sum_{a=1}^{n^2}(\frac{n^2(n^2+1)}{2}) && \text{} \\
&= \sum_{a=1}^{n^2}* n^4 && \text{} \\
&= n^2*n^4 = n^6 && \text{} \\
T(n) &= \Theta(n^6) \\
\end{align*}
\subsection*{1B}
\begin{align*}
T(n) &= \sum_{a=1}^{n^2}(\sum_{b=1}^{n^3}1) && \text{}  \\
&= \sum_{a=1}^{n^2}* n^3 && \text{} \\
&= n^2*n^3 = n^5 && \text{} \\
T(n) &= \Theta(n^5)
\end{align*}
\subsection*{1C}
\begin{align*}
T(n) &= \sum_{a=1}^{n^2}(\sum_{b=1}^{n/5}(\sum_{c=1}^{5b}1)) && \text{}  \\
&= \sum_{a=1}^{n^2}(\sum_{b=1}^{n/5}5b) \\
&= \sum_{a=1}^{n^2}n \\
&= \frac{n^2(n^2+1)}{2} = Cn^4 + ... \\
T(n) &= \Theta(n^4)
\end{align*}
\subsection*{1D}
\begin{align*}
T(n) &= \sum_{a=1}^{n^2}(\sum_{b=1}^{3log_2(a) - 1}1) && \text{}  \\
&= \sum_{a=1}^{n^2}3log_2(a) - 1 \\
&= ((3log_2(n) - 1)(3log_2(n) - 1 + 1))/(2)\\
T(n) &= \Theta(6log(n))
\end{align*}
\subsection*{1E}
\begin{align*}
T(n) &= \sum_{a=1}^{log_3(5n^2)}(\sum_{b=1}^{log_5(10/3)n^3}1) && \text{}  \\
&= \sum_{a=1}^{log_3(5n^2)} log_5(10/3)n^3 && \text{}  \\
&= log_3(5n^2) * log_5((10/3)n^3) \\
T(n) &= \Theta(log(n))
\end{align*}
\subsection*{1F}
\begin{align*}
T(n) &= \sum_{a=1}^{n}(\sum_{b=a}^{n^2}(\sum_{c=1}^{n^3}1)) && \text{}  \\
&= \sum_{a=1}^{n}(\sum_{b=a}^{n^2}* n^3) \\
&= \sum_{a=1}^{n} n^2 * n^3 \\
&= \sum_{a=1}^{n} n^5 \\
&= n *n^5 = n^6 \\
T(n) &= \Theta(n^6)
\end{align*}
\subsection*{1G}
\begin{align*}
T(n) &= \sum_{a=1}^{n/2}(\sum_{b=a}^{a^2}1) && \text{}  \\
&= \sum_{a=1}^{n/2}a^2 && \text{} \\
&= \frac{(n/2)(n/2+1)(2(n/2)+1)}{6} \\
&= \frac{Cn^3 + ...}{6} \\
&= n^3 \\
T(n) &= \Theta(n^3)
\end{align*} 
\subsection*{1H}
\begin{align*}
T(n) &= \sum_{a=1}^{n}1 && \text{} \\
&= n && \text{} \\
T(n) &= \Theta(n)
\end{align*} 



\section*{Problem 2}

% You don't need to write proofs here, just give the expression and evalaute it. The align environment is also good for these kinds of problems.

\subsection*{2A}
\begin{align*}
T(n) &= T(n/2) + 5  && \text{}  \\
T(1) &= 1 && \text{} \\
&= T'(n/4) + n/2 + 5 + 5 \\
&= T'(n/8) + n/4 + n/2 + 5 + 5 + 5\\
T'(n) &= T'(n/(2^k+1)) + 5k\\ 
k &= \log_2(n) - 1 \\
T'(n) &= T'(n/2^{log_2(n)}) + 5(log_2(n) - 1) \\
T'(n) &= n + 5(log_2(n)) - 5 \\
T'(n) &= n + (log_2(n^5)) \\
T'(n) &= \Theta(n)
\end{align*} 
\subsection*{2B}
\begin{align*}
T(n) &= T(n/2) + n  && \text{}  \\
T(1) &= 1 && \text{} \\
&= T'(n/4) + n/2 + n \\
&= T'(n/8) + n/4 + n/2 + n \\
T'(n) &= T'(n/(2^k+1)) + n \\ 
k &= \log_2(n) - 1 \\
T'(n) &= T'(n/2^{log_2(n)}) + n \\
 &= n + n = 2n \\
 T(n) &= \Theta(n)
\end{align*} 
\end{document}
